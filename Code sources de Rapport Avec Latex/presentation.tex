\chapter{Présentation du projet}
\vspace{5cm}
\large{Ce premier chapitre présente tout d’abord d’une manière générale le projet et problématique liée à la résolution des problèmes et les déférentes raisons du choix de ce sujet de projet.\\}


\newpage
\section{Amenant}
\large{Au cours de notre formation en tant qu'Elève Ingénieur de l'ENSIAS en 2ème année, nous sommes appelés à travailler sur un projet Java EE à travers lequel nous exploitons nos connaisances et compétences acquis durant notre formation Développement et Ingéniere Web afin d'aboutir à une application Web basé complètement sur Java EE bien construite.\\
Avec l'accord de notre cher encadrant, nous avons choisi La Location de voiture comme sujet du projet.\\}
\section{Analyse de l'existant}
\large{De nos jours, l’optimisation du temps et la bonne gestion de nos problèmes joue un rôle très important dans notre vie quotidienne afin d’escamoter beaucoup de problèmes et face à la défaillance du système de transport ,le parc automobile marocain croît tous les ans avec son lot de désagréments : embouteillages, pollution, accident de la route. La location de voiture pourrait être la solution à tous ces soucis en réduisant le trafic, et en diminuant la pollution ainsi que la consommation d'énergie et la contribution des délais.\\

Ainsi au Maroc, les gens aujourd'hui optent pour cette alternative. Pour pallier aux problèmes liés à l'insuffisance de l'offre en matière du transport, à l'intérieur ou l'extérieur des villes, les Marocains prennent leurs maux en patience et innovent. Au sein des villes marocaines, face à l'anarchie régnante au sein du secteur du transport.\\

Au Maroc, avec l'émergence des technoligies ,les platformes  destinés à ce nouveau mode de transport sont très peu. Des centaines d'offres et de demandes sont publiées quotidiennement par les internautes, démontrant ainsi un réel engouement envers ce mode de transport destiné à pallier aux retards des trains, manque des moyens pour se procurer une voiture ou encore l'absence de confort de certains autocars.\\}
\section{Problématique soulevée}
La location de voiture n'est pas sans mésaventure. La location de voiture au Maroc échappe à toute surveillance. 
Par conséquent, deux problèmes majeurs trouvent chemin concernant la location du voiture au Maroc :\\

• \textbf{Relation entre agences et locateurs :}\\

Mettre en relation les agences de location et les locateurs et donner beaucoup d'information sur la voiture et l'agences.\\

• \textbf{L'organisation :}\\

Toute personne souhaitant reserver une voiture doit pouvoir trouver facilement grace a quelques cliques son but sans avoir à parcourir plusieurs pages.\\

\section{Solution proposée}
De ce qui précède, nous nous sommes aperçus que le besoin du palatforme de gestion de location de voiture au Maroc, étant un besoin nécessaire et important, est à saisir et mérite une profonde réfexion aux problèmes cités avant. Ainsi, nous l'avons associé à notre projet JEE pour aboutir une application Web dont les objectifs sont :\\

• \textbf{Assurer la location de voiture :}\\

mettre en relation les agences de location et les locateurs.\\


• \textbf{Assurer l'organisation :}\\

établir une app Web intuitive qui va interagir avec le visiteur en toute simplicité sans avoir à parcourir plusieurs pages, seulement avec quelques cliques .\\
