\chapter*{Conclusion générale}
\addcontentsline{toc}{chapter}{Conclusion générale}
Tout au long de ce rapport, nous avons présenté les différentes étapes de la réalisation de l’application du projet. Pour le développement de ce projet nous avons utilisé le langage UML, ce qui a permis de mener correctement la tâche d’analyse des besoins à l’aide du diagramme de cas d’utilisation et la tâche de conception, ainsi les scénarios sont aussi détaillés afin d’expliquer toutes les tâches faites puisque nous travaillons avec la technologie JEE.\\

Ce projet nous a permis de développer nos compétences techniques, d’approfondir
nos connaissances théoriques et pratiques, de stimuler un esprit d’initiative et de
créativité, et notamment dans le domaine de développement des applications web.
Il nous a donné la méthode pour assurer un travail de groupe, comment compter sur
soi pour résoudre les problèmes au cas où ils se présentent, comment être professionnels dans notre travail, comment être attentifs aux indications de notre encadrant, comment être bien organisés pour accomplir dans les meilleurs délais, et meilleures conditions les tâches qui nous sont confiées. Ce projet nous a donné l’occasion de faire le lien entre les connaissances académiques, notamment en JAVA, Base de Données et le monde professionnel.\\

Notre application peut être aisément améliorée grâce à son aspect ouvert. Dans notre application, nous avons travaillé dans un réseau local. Pour la mettre en ligne nous avons seulement besoin de l’héberger sur un serveur.\\